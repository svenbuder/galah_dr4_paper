\begin{table}
\centering
\caption{\textbf{Overlap of GALAH DR4 with \textit{Gaia} DR3 variability catalogues.} Classifications are taken from the \textit{gaiadr3.vari\_classifier\_result} table and described in \cite{Rimoldini2023}.}
\label{tab:overlap_gaiadr3_vari}
\begin{tabular}{ccc}
\hline \hline
Classification & DR4 allspec & DR4 allstar \\
\hline
ACV\vert CP\vert MCP\vert ROAM\vert ROAP\vert SXARI & 330 & 285 \\
ACYG & 45 & 34 \\
BCEP & 16 & 15 \\
BE\vert GCAS\vert SDOR\vert WR & 67 & 54 \\
CEP & 133 & 103 \\
CV & 3 & 3 \\
DSCT\vert GDOR\vert SXPHE & 16342 & 14477 \\
ECL & 4610 & 4074 \\
ELL & 95 & 11 \\
EP & 19 & 14 \\
LPV & 7351 & 6247 \\
RCB & 4 & 1 \\
RR & 561 & 401 \\
RS & 3950 & 3355 \\
S & 40 & 33 \\
SDB & 6 & 6 \\
SOLAR\_LIKE & 20027 & 17256 \\
SPB & 31 & 24 \\
SYST & 3 & 3 \\
WD & 1 & 1 \\
YSO & 1534 & 1096 \\
\hline
Total & 55168 & 47493 \\
\hline
\end{tabular}
\tablefootnote{Abbreviations: ACV ($\alpha$2 Canum Venaticorum), MCP and CP ((magnetic) chemically peculiar), ROAM and ROAP (rapidly oscillating Am- and Ap-type), SXARI (SXArietis), ACYG ($\alpha$ Cygni-type), BCEP ($\beta$ Cephei), BE (B-type emission line), GCAS ($\gamma$ Cassiopeiae), SDOR (SDoradus), WR (Wolf-Rayet), CEP (Cepheid, including anomalous Cepheid, BL Herculis variable, W Virginis variable, $\delta$ Cephei star, RV Tauri-type star, and generic type II Cepheids), CV (cataclysmic variable, excluding supernova and sym- biotic star), DSCT ($\delta$Scuti), GDOR ($\gamma$Doradus), SXPHE (SXPhoenicis), ECL (eclipsing binary including Algol type, $\beta$ Lyrae type, and W Ursae Majoris type), ELL (ellipsoidal variable), EP (star with exoplanet transits), LPV (long-period variable including long secondary period variable, Mira (oCeti) type, Mira or semi- regular  variable, OGLE small amplitude red giant, small amplitude red giant, and semi-regular variable of sub-types), RCB (R Coronae Borealis variable), RR (RR Lyrae star, including fundamental-mode, first overtone, double-mode and anomalous double-mode), RS (RS Canum Venaticorum variable), S (short-timescale object), SDB (subdwarf B star of type V1093 Herculis or V361 Hydrae), SOLAR\\_LIKE (solar-like, including BY Draconis type, rotating spotted star, and flaring stars), SPB (slowly pulsating B-type variable), SYST (symbiotic variable star, including Z Andromedae type), WD (variable white dwarf), YSO (young stellar object (YSO), including dipper stars, eruptive YSOs such as FU Orionis type variables, pulsating pre-main-sequence stars, Herbig Ae or Be types, including UX Orionis stars, and T Tauri stars).}
\end{table}
